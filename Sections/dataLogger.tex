\section{Data logger}
\label{sec:dataLogger}

The x-IMU3 can function as a stand-alone data logger by streaming real-time data to a file on the \ac{microSD}.  Files created by the data logger use the .ximu3 extension and can be downloaded from the x-IMU3 to be converted to \ac{CSV} files using the x-IMU3 software.

The data logger will create a new file in the \enquote{Data Logger} directory on the \ac{microSD} each time logging starts.  Files will never be overwritten or deleted by the data logger.  If the \ac{microSD} becomes full then the data logger will stop and the x-IMU3 will indicate an error.

\subsection{Automatic start stop}
\label{sec:automaticStartStop}

Automatic start stop can be enabled or disabled in the device settings.  If automatic start stop is enabled then the data logger will start when the x-IMU3 is switched on and stop when the x-IMU3 is switched off.  The data logger will also stop when the \ac{microSD} is accessed by \ac{USB} or Wi-Fi.

\subsubsection{File name}
\label{sec:fileName}

The file name is a combination of a prefix and either a counter or the date and time. The prefix is configured, and the counter is enabled or disabled in the device settings.  If the counter is enabled then the file name format is \enquote{prefix\_CCCC.ximu3} where \enquote{CCCC} is a four digit number between 0000 and 9999 that increments each time it is used.  If the file name already exists then the counter will increment until the file name is available.  Incrementing beyond 9999 will reset the counter to 0000.

If the counter is disabled then the file name format is \enquote{prefix\_YYYY-MM-DD\_hh-mm-ss.ximu3}, indicating the the date and time that the file was created.  For example, if the prefix is \enquote{Test} then a file created at 3.30 p.m. on January 20, 2025 would have the name \enquote{Test\_2025-01-20\_15-30-00.ximu3}.

The prefix should not include spaces and should only use alphanumeric characters, hyphens, periods, or underscores.  Invalid characters will not be included in the file name.  The prefix may be left blank to simplify the file name format to either \enquote{CCCC.ximu3} or \enquote{YYYY-MM-DD\_hh-mm-ss.ximu3}

\subsubsection{Maximum file size and period}
\label{sec:maximumFileSizeAndPeriod}

A maximum file size and maximum file period can be configured in the device settings.  The data logger will start a new file each time the file size reaches the maximum file size, or the period since the start of the file reaches the maximum file period.

\subsection{Start and stop commands}

The start and stop commands provide manual control of the data logger when automatic start stop is disabled.  See \Fref{sec:startCommand} and \Fref{sec:stopCommand} for a description of each command.

\subsection{File contents}

The contents of a file is a byte stream as per the communication protocol described in \Fref{sec:communicationProtocol}.  Each file starts with a preamble of the following messages, in order.

\begin{enumerate}[nolistsep]
    \item Ping response
    \item Write time command
    \item Write setting command for each setting
\end{enumerate}

\subsection{Downloading files}

Data logger files can be download by \ac{USB} or Wi-Fi.  The \ac{microSD} appears as an external drive when \ac{USB} is connected and can be accessed by Wi-Fi by typing in the x-IMU3's \ac{IP} address to a browser.  The \ac{microSD} cannot be accessed while the data logger is active.
