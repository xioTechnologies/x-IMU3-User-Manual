\section{Communication protocol}
\label{sec:communicationProtocol}

All communication interfaces use the same communication protocol.  The byte stream is therefore identical for \ac{USB}, serial, \ac{TCP}, \ac{UDP}, Bluetooth, and the files created by the data logger.  The communication protocol consists of two message types:

\begin{itemize}[nolistsep]
    \item Command messages
    \item Data messages
\end{itemize}

All messages are terminated by a \ac{LF} control character.  This termination byte will not appear anywhere else in a message and so can be used to divide a byte stream into individual messages.  \Fref{tab:controlCharactersLFrepresentations} describes the different ways that the control character \ac{LF} may be referred to throughout this document.

\customTable
{l c c c c}
{Control character & Abbreviation & String & Hex & Decimal}
{
    \acl{LF} & \acs{LF} & \enquote{\textbackslash n} & 0x0A & 10\\
}
{Control characters \acs{LF} representations}
{tab:controlCharactersLFrepresentations}

The first byte of a message indicates the message type.  Command messages start with the character \enquote{\{} (0x7B in hex, 123 in decimal).  Data messages start with either an uppercase character or a byte value greater than 0x80 (128 in decimal) depending on the message.

\subsection{Command messages}

Command messages are sent to the x-IMU3 to read and write settings and execute commands.  All command messages are a \ac{JSON} object containing a single key/value pair, terminated by the control character \ac{LF}.  The control character \ac{LF} must not appear anywhere else in a command message.  The x-IMU3 will acknowledge each received command message by sending a command message with the same key to the host.

The key used by command messages sent to the x-IMU3 is not case sensitive and can use non-alphanumeric characters arbitrarily.  For example, \enquote{serialNumber}, \enquote{Serial Number}, and \enquote{serial\_number} are all valid keys for a command message to read the device serial number.

\newcommand{\commandMessageExample}[2]{
    \begin{table}[H]
        \def\arraystretch{1.5}
        \begin{tabular}{l l}
            \textbf{Example:} & \texttt{\{#1\}\textbackslash n}
        \end{tabular}
    \end{table}
}

\subsubsection{Read setting command}

The read setting command is sent to the x-IMU3 to read a setting value.  The key is the setting key and the value is null.  See \Fref{sec:individualSettings} for a complete list of settings.  The x-IMU3 will acknowledge a read setting command by sending a write setting command to the host.

\commandMessageExample{"serialNumber":null}

\subsubsection{Write setting command}

The write setting command is sent to the x-IMU3 to write a setting value, or sent from the x-IMU3 to the host in response to a read setting command.  The key is the setting key and the value is the setting value.  See \Fref{sec:individualSettings} for a complete list of settings.  The x-IMU3 will acknowledge a write setting command by sending a setting write command back to the host, indicating the new settings value.  The x-IMU3 will not apply new settings until two seconds after the most recent write setting command or default command was received.

\commandMessageExample{"deviceName":"x-IMU3"}

\subsubsection{Default command}

The default command is sent to the x-IMU3 to set all settings to default values.  The key is \enquote{default} and the value is null.  The x-IMU3 will not apply new settings until two seconds after the most recent write setting command or default command was received.

\commandMessageExample{"default":null}

\subsubsection{Apply command}

The apply command is sent to the x-IMU3 to apply all settings.  The key is \enquote{apply} and the value is null.  This command can be sent after a write setting or default command to apply settings immediately instead of after a two second delay.

\commandMessageExample{"apply":null}

\subsubsection{Save command}

The save command is sent to the x-IMU3 to save all settings to \ac{EEPROM}.  The key is \enquote{save} and the value is null.  The command acknowledgement will not be sent until the save is complete.  This may take up to 300 milliseconds.  The save command is unnecessary in most applications because the x-IMU3 will automatically save all settings on shutdown.

\commandMessageExample{"save":null}

\subsubsection{Read time command}

The read time command is sent to the x-IMU3 to read the date and time of the \ac{RTC}.  The key is \enquote{time} and the value is null.  The x-IMU3 will acknowledge a read time command by sending a write time command to the host.

\commandMessageExample{"time":null}

\subsubsection{Write time command}

The write time command is sent to the x-IMU3 to write the date and time of the \ac{RTC}, or sent from the x-IMU3 to the host in response to a read time command.  The key is \enquote{time} and the value is a string expressing the date and time in the format \enquote{YYYY-MM-DD hh:mm:ss} where each delimiter can be any non-numerical character.  The x-IMU3 will acknowledge a write time command by sending a write time command back to the host, indicating the new date and time.

\commandMessageExample{"time":"2020-01-01 00:00:00"}

\subsubsection{Ping command}

The ping command is sent to the x-IMU3 to trigger a ping response.  The key is \enquote{ping} and the value is null.  The x-IMU3 will acknowledge a ping command by sending a ping response to the host.

\commandMessageExample{"ping":null}

\subsubsection{Ping response}

The ping response is sent from the x-IMU3 to the host in response to the ping command.  The key is \enquote{ping} and the value is a \ac{JSON} object containing three key/value pairs indicating the communication interface, device name, and device serial number.  The keys are \enquote{interface}, \enquote{deviceName}, and \enquote{serialNumber}, respectively and all values are string types.

\begin{table}[H]
    \begin{tabular}{l l l}
        \textbf{Example}*\textbf{:} & \texttt{\{}\\
        & \texttt{~~"ping":~\{} &\\
        & \texttt{~~~~"interface":} & \texttt{"USB",}\\
        & \texttt{~~~~"deviceName":} & \texttt{"x-IMU3",}\\
        & \texttt{~~~~"serialNumber":} & \texttt{"01234567"}\\
        & \texttt{~~\}}\\
        & \texttt{\}\textbackslash n}\\
    \end{tabular}\\
    \begin{tabular}{l}
        \\
        \footnotesize{* The actual \acs{JSON} will not include any whitespace.}
    \end{tabular}
\end{table}

\subsubsection{Reset command}

The reset command is sent to the x-IMU3 to reset the x-IMU3.  The key is \enquote{reset} and the value is null.  A reset is equivalent to switching the x-IMU3 off and then on again.  The x-IMU3 will reset two seconds after receiving this command.

\commandMessageExample{"reset":null}

\subsubsection{Shutdown command}

The shutdown command is sent to the x-IMU3 to switch the x-IMU3 off.  The key is \enquote{shutdown} and the value is null.  The x-IMU3 will shutdown two seconds after receiving this command.

\commandMessageExample{"shutdown":null}

\subsubsection{Strobe command}
\label{sec:strobeCommand}

The strobe command is sent to the x-IMU3 to strobe the \ac{LED} bright white for 5 seconds.  The key is \enquote{strobe} and the value is null.  This command can be used to quickly locate a specific x-IMU3 when using multiple x-IMU3s.

\commandMessageExample{"strobe":null}

\subsubsection{Colour command}
\label{sec:colourCommand}

The colour command is sent to the x-IMU3 to set the \ac{LED} colour.  The key is \enquote{colour} or \enquote{color} and the value is either a \ac{RGB} hex triplet expressed as a string, or null.  Setting the colour will override the normal \ac{LED} behaviour.  A value of null will restore the normal behaviour.

\commandMessageExample{"colour":"FFFFFF"}

\subsubsection{Initialise \acs{AHRS} command}

The initialise \ac{AHRS} command is sent to the x-IMU3 to reinitialise the \ac{AHRS} algorithm.  The key is \enquote{initialise} or \enquote{initialize} and the value is null.  This command in unnecessary for normal operation and typically only used during calibration and testing.

\commandMessageExample{"initialise":null}

\subsubsection{Heading command}

The heading command is sent to the x-IMU3 to set the heading of the orientation measurement provided by the \ac{AHRS} algorithm.  The key is \enquote{heading} and the value is a number equal to the heading in degrees.  The heading command can only be used if the magnetometer is ignored in the \ac{AHRS} settings.

\commandMessageExample{"heading":0}

\subsubsection{Serial accessory command}

The serial accessory command is sent to the x-IMU3 to transmit data to a serial accessory when the serial interface is in serial accessory mode.  The key is \enquote{accessory} and the value is the data expressed as a string of up to 256 characters.  Longer strings will be truncated to the maximum size.  The string escape sequence \enquote{\textbackslash u} can be used to express any byte value as per the \ac{JSON} specification.

\commandMessageExample{"accessory":"hello \textbackslash u0077\textbackslash u006F\textbackslash u0072\textbackslash u006C\textbackslash u0064"}

\subsubsection{Note command}

The note command is sent to the x-IMU3 to generate a timestamped notification message containing a user-defined string.  The key is \enquote{note} and the value is the string of up to 127 characters.  Longer strings will be truncated to the maximum size.  This command can be used to create timestamped notes of events during data logging.

\commandMessageExample{"note":"Something happened."}

\subsubsection{Format command}

The format command is sent to the x-IMU3 to format the \ac{microSD}.  The key is \enquote{format} and the value is null.  The command acknowledgement will not be sent until the format is complete.  This will take approximately 3 seconds for an 8 GB \ac{microSD}.  Larger SD \acp{microSD} will take longer to format.  Formatting will erase all data on the SD card.

\commandMessageExample{"format":null}

\subsubsection{Self-test command}

The self-test command is sent to the x-IMU3 to perform a self-test.  The key is \enquote{test} and the value is null.  The x-IMU3 will acknowledge a self-test command by sending a self-test response to the host once the self-test is complete.  This may take up to 5 seconds.  The x-IMU3 must be stationary during the self-test.

\commandMessageExample{"test":null}

\subsubsection{Self-test response}

The self-test response is sent from the x-IMU3 to the host in response to the self-test command.  The key is \enquote{test} and the value is a \ac{JSON} object containing multiple key/value pairs.  Each key/value pair indicates the result of a test.  Each key is the test name and the value is the string "Passed" if the test was passed.

\begin{table}[H]
    \begin{tabular}{l l l}
        \textbf{Example}*\textbf{:} & \texttt{\{}\\
        & \texttt{~~"test":~\{} &\\
        & \texttt{~~~~"EEPROM":} & \texttt{"Passed",}\\
        & \texttt{~~~~"RTC":} & \texttt{"Passed",}\\
        & \texttt{~~~~"Inertial":} & \texttt{"Passed",}\\
        & \texttt{~~~~"Magnetometer":} & \texttt{"Passed",}\\
        & \texttt{~~~~"High-g Accelerometer":} & \texttt{"Passed",}\\
        & \texttt{~~~~"Battery":} & \texttt{"Passed",}\\
        & \texttt{~~~~"SD Card":} & \texttt{"Passed",}\\
        & \texttt{~~~~"Wireless":} & \texttt{"Passed",}\\
        & \texttt{~~\}}\\
        & \texttt{\}\textbackslash n}\\
    \end{tabular}\\
    \begin{tabular}{l}
        \\
        \footnotesize{* The actual \acs{JSON} will not include any whitespace.}
    \end{tabular}
\end{table}

\subsubsection{Bootloader command}

The bootloader command is sent to the x-IMU3 to put the x-IMU3 in bootloader mode.  The key is \enquote{bootloader} and the value is null.  The x-IMU3 will enter bootloader mode two seconds after receiving this command.

\commandMessageExample{"bootloader":null}

\subsubsection{Factory command}

The factory command is sent to the x-IMU3 to enable factory mode.  The key is \enquote{factory} and the value is null.  In factory mode, read-only settings can be written using the write setting command and the erase command will be enabled.

\commandMessageExample{"factory":null}

\newcommand{\commandWarning}{\warning{Incorrect use of this command may permanently damage the device.  Do not use this command unless instructed by customer support.}}

\commandWarning

\subsubsection{Erase command}

The erase command is sent to the x-IMU3 to erase the \ac{EEPROM}.  The key is \enquote{erase} and the value is null.  The command acknowledgement will not be sent until the erase is complete.  This will take approximately 700 milliseconds.  This command can only be used if factory mode is enabled.

\commandMessageExample{"erase":null}

\commandWarning

\subsection{Data messages}
\label{sec:dataMessages}

Data messages are sent from the x-IMU3 to the host to provide timestamped measurements, serial accessory data, notifications, and error messages.  Data messages will be either \ac{ASCII} or binary, depending on the device settings.

\ac{ASCII} data messages consist of multiple comma-separated values terminated by the control character character \ac{LF}.  The first value is a single uppercase character indicating the message type.  The second value is the timestamp in microseconds.  The remaining values are arguments specific to the message type.

Binary data messages are a sequence of bytes terminated by the control character \ac{LF}.  The first byte of the sequence indicates the message type.  The value of this byte is equal to 0x80 plus the first character of the equivalent \ac{ASCII} message.  The next eight bytes are the timestamp in microseconds expressed as a 64-bit unsigned integer.  The remaining bytes are arguments specific to the message type.  Numerical types use little-endian ordering.  Byte stuffing is used to remove all occurrences of the control character \ac{LF} prior to the termination byte.

\subsubsection{Byte stuffing}
\label{sec:byteStuffing}

Byte stuffing ensures that the termination byte value, 0x0A, only occurs at the end of a binary data message.  This is achieved by replacing all occurrences of the termination byte prior to termination with an escape sequence.  This process is identical to \ac{SLIP} except that the \enquote{END} byte value is defined as 0x0A.  \Fref{tab:valuesUsedByTheByteStuffingProcess} lists the values used by the byte stuffing process.

\customTable
{l l l l}
{Hex & Decimal & Name & Description}
{
0x0A & 10 & END & Message termination\\
0xDB & 219 & ESC & Message escape\\
0xDC & 220 & ESC\_END & Transposed message termination\\
0xDD & 221 & ESC\_ESC & Transposed message escape\\
}
{Values used by the byte stuffing process}
{tab:valuesUsedByTheByteStuffingProcess}

Byte stuffing is achieved by the following:

\begin{itemize}
    \item Replace each occurrence of END in the original message with the two byte sequence: ESC, ESC\_END.
    \item Replace each occurrence of ESC in the original message with the two byte sequence: ESC, ESC\_ESC.
\end{itemize}

The byte stuffing process will not modify the END that terminates the message.  \Fref{tab:byteStuffingExamples} demonstrates byte stuffing for example byte sequences terminated as binary data messages.

\begingroup
    \definecolor{colourA}{HTML}{1F77B4} % Tableau colours
    \definecolor{colourB}{HTML}{FF7F0E}
    \definecolor{colourC}{HTML}{2CA02C}
    \customTable
    {c l l}
    {Example & Before byte stuffing & After byte stuffing}
    {
    1 & \texttt{45 58 41 4D 50 4C 45 \textcolor{colourC}{0A}} & \texttt{45 58 41 4D 50 4C 45 \textcolor{colourC}{0A}}\\
    2 & \texttt{45 \textcolor{colourA}{0A} 41 4D 50 4C 45 \textcolor{colourC}{0A}} & \texttt{45 \textcolor{colourA}{DB DC} 41 4D 50 4C 45 \textcolor{colourC}{0A}}\\
    3 & \texttt{45 58 \textcolor{colourB}{DB} 4D 50 4C 45 \textcolor{colourC}{0A}} & \texttt{45 58 \textcolor{colourB}{DB DD} 4D 50 4C 45 \textcolor{colourC}{0A}}\\
    4 & \texttt{45 58 41 4D 50 \textcolor{colourB}{DB} \textcolor{colourA}{0A} \textcolor{colourC}{0A}} & \texttt{45 58 41 4D 50 \textcolor{colourB}{DB DD} \textcolor{colourA}{DB DC} \textcolor{colourC}{0A}}\\
    }
    {Byte stuffing examples}
    {tab:byteStuffingExamples}
\endgroup

\newcommand{\dataMessageTable}[2]{
    \customTable
    {c l}
    {Argument & Description}
    {
        \ifdefined\tempArgumentA 1 & \tempArgumentA\\ \fi
        \ifdefined\tempArgumentB 2 & \tempArgumentB\\ \fi
        \ifdefined\tempArgumentC 3 & \tempArgumentC\\ \fi
        \ifdefined\tempArgumentD 4 & \tempArgumentD\\ \fi
        \ifdefined\tempArgumentE 5 & \tempArgumentE\\ \fi
        \ifdefined\tempArgumentF 6 & \tempArgumentF\\ \fi
        \ifdefined\tempArgumentG 7 & \tempArgumentG\\ \fi
        \ifdefined\tempArgumentH 8 & \tempArgumentH\\ \fi
        \ifdefined\tempArgumentI 9 & \tempArgumentI\\ \fi
    }
    {#1}
    {#2}
}

\newcommand{\dataMessageExample}{
    The following message examples are for a timestamp of 1 second (1,000,000 microseconds) and argument values of:

    \definecolor{colourA}{HTML}{1F77B4} % Tableau 10 colours
    \definecolor{colourB}{HTML}{FF7F0E}
    \definecolor{colourC}{HTML}{2CA02C}
    \definecolor{colourD}{HTML}{D92728}
    \definecolor{colourE}{HTML}{9482BD}
    \definecolor{colourF}{HTML}{8C564B}
    \definecolor{colourG}{HTML}{E377C2}
    \definecolor{colourH}{HTML}{7F7F7F}
    \definecolor{colourI}{HTML}{BCBD22}
    %\definecolor{colourJ}{HTML}{17BECF}

    \begin{enumerate}[nolistsep]
        \ifdefined\tempNameA \item {\tempNameA} = \textcolor{colourA}{\tempValueA}\fi
        \ifdefined\tempNameB \item {\tempNameB} = \textcolor{colourB}{\tempValueB}\fi
        \ifdefined\tempNameC \item {\tempNameC} = \textcolor{colourC}{\tempValueC}\fi
        \ifdefined\tempNameD \item {\tempNameD} = \textcolor{colourD}{\tempValueD}\fi
        \ifdefined\tempNameE \item {\tempNameE} = \textcolor{colourE}{\tempValueE}\fi
        \ifdefined\tempNameF \item {\tempNameF} = \textcolor{colourF}{\tempValueF}\fi
        \ifdefined\tempNameG \item {\tempNameG} = \textcolor{colourG}{\tempValueG}\fi
        \ifdefined\tempNameH \item {\tempNameH} = \textcolor{colourH}{\tempValueH}\fi
        \ifdefined\tempNameI \item {\tempNameI} = \textcolor{colourI}{\tempValueI}\fi
    \end{enumerate}

    \begin{table}[H]
        \def\arraystretch{1.5}
        \begin{tabularx}{\linewidth}{l X}
            \textbf{\acs{ASCII} example:} &
            \texttt{\tempAsciiFirst,1000000}%
            \ifdefined\tempNameA \texttt{,\textcolor{colourA}{\tempAsciiA}}\fi
            \ifdefined\tempNameB \texttt{,\textcolor{colourB}{\tempAsciiB}}\fi
            \ifdefined\tempNameC \texttt{,\textcolor{colourC}{\tempAsciiC}}\fi
            \ifdefined\tempNameD \texttt{,\textcolor{colourD}{\tempAsciiD}}\fi
            \ifdefined\tempNameE \texttt{,\textcolor{colourE}{\tempAsciiE}}\fi
            \ifdefined\tempNameF \texttt{,\textcolor{colourF}{\tempAsciiF}}\fi
            \ifdefined\tempNameG \texttt{,\textcolor{colourG}{\tempAsciiG}}\fi
            \ifdefined\tempNameH \texttt{,\textcolor{colourH}{\tempAsciiH}}\fi
            \ifdefined\tempNameI \texttt{,\textcolor{colourI}{\tempAsciiI}}\fi
            \texttt{\textbackslash n}\\
            \textbf{Binary example:} &
            \texttt{{\tempBinaryFirst} 40 42 0F 00 00 00 00 00 }%
            \ifdefined\tempNameA \texttt{\textcolor{colourA}{\tempBinaryA} }\fi
            \ifdefined\tempNameB \texttt{\textcolor{colourB}{\tempBinaryB} }\fi
            \ifdefined\tempNameC \texttt{\textcolor{colourC}{\tempBinaryC} }\fi
            \ifdefined\tempNameD \texttt{\textcolor{colourD}{\tempBinaryD} }\fi
            \ifdefined\tempNameE \texttt{\textcolor{colourE}{\tempBinaryE} }\fi
            \ifdefined\tempNameF \texttt{\textcolor{colourF}{\tempBinaryF} }\fi
            \ifdefined\tempNameG \texttt{\textcolor{colourG}{\tempBinaryG} }\fi
            \ifdefined\tempNameH \texttt{\textcolor{colourH}{\tempBinaryH} }\fi
            \ifdefined\tempNameI \texttt{\textcolor{colourI}{\tempBinaryI} }\fi
            \texttt{0A}\\
        \end{tabularx}
    \end{table}
}

\subsubsection{Inertial message}

The inertial message provides timestamped gyroscope and accelerometer measurements.  Inertial messages are sent continuously at the message rate configured in the device settings.  The first value of an \ac{ASCII} message is the character \enquote{I} and the arguments are six numerical values expressed to four decimal places.  The first byte of a binary message is 0xC9 (equal to 0x80 + \enquote{I}) and the arguments are six contiguous 32-bit floats.  The message arguments are described in \Fref{tab:inertialMessageArguments}.

\begingroup
    \def\tempArgumentA{Gyroscope X axis in degrees per second}
    \def\tempArgumentB{Gyroscope Y axis in degrees per second}
    \def\tempArgumentC{Gyroscope Z axis in degrees per second}
    \def\tempArgumentD{Accelerometer X axis in g}
    \def\tempArgumentE{Accelerometer Y axis in g}
    \def\tempArgumentF{Accelerometer Z axis in g}
    \dataMessageTable
    {Inertial message arguments}
    {tab:inertialMessageArguments}
\endgroup

\begingroup
    \def\tempNameA{Gyroscope X axis}
    \def\tempNameB{Gyroscope Y axis}
    \def\tempNameC{Gyroscope Z axis}
    \def\tempNameD{Accelerometer X axis}
    \def\tempNameE{Accelerometer Y axis}
    \def\tempNameF{Accelerometer Z axis}
    \def\tempValueA{0}
    \def\tempValueB{0}
    \def\tempValueC{0}
    \def\tempValueD{0}
    \def\tempValueE{0}
    \def\tempValueF{1}
    \def\tempAsciiFirst{I}
    \def\tempAsciiA{0.0000}
    \def\tempAsciiB{0.0000}
    \def\tempAsciiC{0.0000}
    \def\tempAsciiD{0.0000}
    \def\tempAsciiE{0.0000}
    \def\tempAsciiF{1.0000}
    \def\tempBinaryFirst{C9}
    \def\tempBinaryA{00 00 00 00}
    \def\tempBinaryB{00 00 00 00}
    \def\tempBinaryC{00 00 00 00}
    \def\tempBinaryD{00 00 00 00}
    \def\tempBinaryE{00 00 00 00}
    \def\tempBinaryF{00 00 80 3F}
    \dataMessageExample
\endgroup

\subsubsection{Magnetometer message}

The magnetometer message provides timestamped magnetometer measurements.  Magnetometer messages are sent continuously at the message rate configured in the device settings.  The first value of an \ac{ASCII} message is the character \enquote{M} and the arguments are three numerical values expressed to four decimal places.  The first byte of a binary message is 0xCD (equal to 0x80 + \enquote{M}) and the arguments are three contiguous 32-bit floats.  The message arguments are described in \Fref{tab:magnetometerMessageArguments}.

\begingroup
    \def\tempArgumentA{Magnetometer X axis in \acs{a.u.}}
    \def\tempArgumentB{Magnetometer Y axis in \acs{a.u.}}
    \def\tempArgumentC{Magnetometer Z axis in \acs{a.u.}}
    \dataMessageTable
    {Magnetometer message arguments}
    {tab:magnetometerMessageArguments}
\endgroup

\begingroup
    \def\tempNameA{Magnetometer X axis}
    \def\tempNameB{Magnetometer Y axis}
    \def\tempNameC{Magnetometer Z axis}
    \def\tempValueA{1}
    \def\tempValueB{0}
    \def\tempValueC{0}
    \def\tempAsciiFirst{M}
    \def\tempAsciiA{1.0000}
    \def\tempAsciiB{0.0000}
    \def\tempAsciiC{0.0000}
    \def\tempBinaryFirst{CD}
    \def\tempBinaryA{00 00 80 3F}
    \def\tempBinaryB{00 00 00 00}
    \def\tempBinaryC{00 00 00 00}
    \dataMessageExample
\endgroup

\subsubsection{Quaternion message}

The quaternion message provides timestamped measurements of the orientation of the x-IMU3 relative to the Earth.  Quaternion messages are sent continuously at the message rate configured in the device settings.  The first value of an \ac{ASCII} message is the character \enquote{Q} and the arguments are four numerical values expressed to four decimal places.  The first byte of a binary message is 0xD1 (equal to 0x80 + \enquote{Q}) and the arguments are four contiguous 32-bit floats.  The message arguments are described in \Fref{tab:quaternionMessageArguments}.

\begingroup
    \def\tempArgumentA{Quaternion W element}
    \def\tempArgumentB{Quaternion X element}
    \def\tempArgumentC{Quaternion Y element}
    \def\tempArgumentD{Quaternion Z element}
    \def\tempCaption{Quaternion message arguments}
    \def\tempLabel{tab:quaternionMessageArguments}
    \dataMessageTable
    {Quaternion message arguments}
    {tab:quaternionMessageArguments}
\endgroup

\begingroup
    \def\tempNameA{Quaternion W element}
    \def\tempNameB{Quaternion X element}
    \def\tempNameC{Quaternion Y element}
    \def\tempNameD{Quaternion Z element}
    \def\tempValueA{1}
    \def\tempValueB{0}
    \def\tempValueC{0}
    \def\tempValueD{0}
    \def\tempAsciiFirst{Q}
    \def\tempAsciiA{1.0000}
    \def\tempAsciiB{0.0000}
    \def\tempAsciiC{0.0000}
    \def\tempAsciiD{0.0000}
    \def\tempBinaryFirst{D1}
    \def\tempBinaryA{00 00 80 3F}
    \def\tempBinaryB{00 00 00 00}
    \def\tempBinaryC{00 00 00 00}
    \def\tempBinaryD{00 00 00 00}
    \dataMessageExample
\endgroup

\subsubsection{Rotation matrix message}

The rotation matrix message provides timestamped measurements of the orientation of the x-IMU3 relative to the Earth.  Rotation matrix messages are sent continuously at the message rate configured in the device settings.  The first value of an \ac{ASCII} message is the character \enquote{R} and the arguments are nine numerical values expressed to four decimal places.  The first byte of a binary message is 0xD2 (equal to 0x80 + \enquote{R}) and the arguments are nine contiguous 32-bit floats.  The message arguments are described in \Fref{tab:rotationMatrixMessageArguments}.

\begingroup
    \def\tempArgumentA{Rotation matrix XX element}
    \def\tempArgumentB{Rotation matrix XY element}
    \def\tempArgumentC{Rotation matrix XZ element}
    \def\tempArgumentD{Rotation matrix YX element}
    \def\tempArgumentE{Rotation matrix YY element}
    \def\tempArgumentF{Rotation matrix YZ element}
    \def\tempArgumentG{Rotation matrix ZX element}
    \def\tempArgumentH{Rotation matrix ZY element}
    \def\tempArgumentI{Rotation matrix ZZ element}
    \dataMessageTable
    {Rotation matrix message arguments}
    {tab:rotationMatrixMessageArguments}
\endgroup

\begingroup
    \def\tempNameA{Rotation matrix XX element}
    \def\tempNameB{Rotation matrix XY element}
    \def\tempNameC{Rotation matrix XZ element}
    \def\tempNameD{Rotation matrix YX element}
    \def\tempNameE{Rotation matrix YY element}
    \def\tempNameF{Rotation matrix YZ element}
    \def\tempNameG{Rotation matrix ZX element}
    \def\tempNameH{Rotation matrix ZY element}
    \def\tempNameI{Rotation matrix ZZ element}
    \def\tempValueA{1}
    \def\tempValueB{0}
    \def\tempValueC{0}
    \def\tempValueD{0}
    \def\tempValueE{1}
    \def\tempValueF{0}
    \def\tempValueG{0}
    \def\tempValueH{0}
    \def\tempValueI{1}
    \def\tempAsciiFirst{R}
    \def\tempAsciiA{1.0000}
    \def\tempAsciiB{0.0000}
    \def\tempAsciiC{0.0000}
    \def\tempAsciiD{0.0000}
    \def\tempAsciiE{1.0000}
    \def\tempAsciiF{0.0000}
    \def\tempAsciiG{0.0000}
    \def\tempAsciiH{0.0000}
    \def\tempAsciiI{1.\linebreak0000} % \texttt will not line break
    \def\tempBinaryFirst{D2}
    \def\tempBinaryA{00 00 80 3F}
    \def\tempBinaryB{00 00 00 00}
    \def\tempBinaryC{00 00 00 00}
    \def\tempBinaryD{00 00 00 00}
    \def\tempBinaryE{00 00 80 3F}
    \def\tempBinaryF{00 00 00 00}
    \def\tempBinaryG{00 00 00 00}
    \def\tempBinaryH{00 00 00 00}
    \def\tempBinaryI{00 00 80 3F}
    \dataMessageExample
\endgroup

\subsubsection{Euler angles message}

The Euler angles message provides timestamped measurements of the orientation of the x-IMU3 relative to the Earth.  Euler angles messages are sent continuously at the message rate configured in the device settings.  The first value of an \ac{ASCII} message is the character \enquote{A} and the arguments are three numerical values expressed to four decimal places.  The first byte of a binary message is 0xC1 (equal to 0x80 + \enquote{A}) and the arguments are three contiguous 32-bit floats.  The message arguments are described in \Fref{tab:eulerAnglesMessageArguments}.

\begingroup
    \def\tempArgumentA{Roll angle in degrees}
    \def\tempArgumentB{Pitch angle in degrees}
    \def\tempArgumentC{Yaw angle in degrees}
    \dataMessageTable
    {Euler angles message arguments}
    {tab:eulerAnglesMessageArguments}
\endgroup

\begingroup
    \def\tempNameA{Roll angle}
    \def\tempNameB{Pitch angle}
    \def\tempNameC{Yaw angle}
    \def\tempValueA{0}
    \def\tempValueB{0}
    \def\tempValueC{0}
    \def\tempAsciiFirst{A}
    \def\tempAsciiA{0.0000}
    \def\tempAsciiB{0.0000}
    \def\tempAsciiC{0.0000}
    \def\tempBinaryFirst{C1}
    \def\tempBinaryA{00 00 00 00}
    \def\tempBinaryB{00 00 00 00}
    \def\tempBinaryC{00 00 00 00}
    \dataMessageExample
\endgroup

\subsubsection{Linear acceleration message}

The linear acceleration message provides timestamped measurements of linear acceleration and the orientation of the x-IMU3 relative to the Earth.  Linear acceleration messages are sent continuously at the message rate configured in the device settings.  The first value of an \ac{ASCII} message is the character \enquote{L} and the arguments are seven numerical values expressed to four decimal places.  The first byte of a binary message is 0xCC (equal to 0x80 + \enquote{L}) and the arguments are seven contiguous 32-bit floats.  The message arguments are described in \Fref{tab:linearAccelerationMessageArguments}.

\begingroup
    \def\tempArgumentA{Quaternion W element}
    \def\tempArgumentB{Quaternion X element}
    \def\tempArgumentC{Quaternion Y element}
    \def\tempArgumentD{Quaternion Z element}
    \def\tempArgumentE{Linear acceleration X axis in g}
    \def\tempArgumentF{Linear acceleration Y axis in g}
    \def\tempArgumentG{Linear acceleration Z axis in g}
    \def\tempCaption{Linear acceleration message arguments}
    \def\tempLabel{tab:linearAccelerationMessageArguments}
    \dataMessageTable
    {Linear acceleration message arguments}
    {tab:linearAccelerationMessageArguments}
\endgroup

\begingroup
    \def\tempNameA{Quaternion W element}
    \def\tempNameB{Quaternion X element}
    \def\tempNameC{Quaternion Y element}
    \def\tempNameD{Quaternion Z element}
    \def\tempNameE{Linear acceleration X axis}
    \def\tempNameF{Linear acceleration Y axis}
    \def\tempNameG{Linear acceleration Z axis}
    \def\tempValueA{1}
    \def\tempValueB{0}
    \def\tempValueC{0}
    \def\tempValueD{0}
    \def\tempValueE{0}
    \def\tempValueF{0}
    \def\tempValueG{0}
    \def\tempAsciiFirst{L}
    \def\tempAsciiA{1.0000}
    \def\tempAsciiB{0.0000}
    \def\tempAsciiC{0.0000}
    \def\tempAsciiD{0.0000}
    \def\tempAsciiE{0.0000}
    \def\tempAsciiF{0.0000}
    \def\tempAsciiG{0.0000}
    \def\tempBinaryFirst{CC}
    \def\tempBinaryA{00 00 80 3F}
    \def\tempBinaryB{00 00 00 00}
    \def\tempBinaryC{00 00 00 00}
    \def\tempBinaryD{00 00 00 00}
    \def\tempBinaryE{00 00 00 00}
    \def\tempBinaryF{00 00 00 00}
    \def\tempBinaryG{00 00 00 00}
    \dataMessageExample
\endgroup

\subsubsection{Earth acceleration message}

The Earth acceleration message provides timestamped measurements of Earth acceleration and the orientation of the x-IMU3 relative to the Earth.  Earth acceleration messages are sent continuously at the message rate configured in the device settings.  The first value of an \ac{ASCII} message is the character \enquote{E} and the arguments are seven numerical values expressed to four decimal places.  The first byte of a binary message is 0xC5 (equal to 0x80 + \enquote{E}) and the arguments are seven contiguous 32-bit floats.  The message arguments are described in \Fref{tab:earthAccelerationMessageArguments}.

\begingroup
    \def\tempArgumentA{Quaternion W element}
    \def\tempArgumentB{Quaternion X element}
    \def\tempArgumentC{Quaternion Y element}
    \def\tempArgumentD{Quaternion Z element}
    \def\tempArgumentE{Earth acceleration X axis in g}
    \def\tempArgumentF{Earth acceleration Y axis in g}
    \def\tempArgumentG{Earth acceleration Z axis in g}
    \def\tempCaption{Earth acceleration message arguments}
    \def\tempLabel{tab:earthAccelerationMessageArguments}
    \dataMessageTable
    {Earth acceleration message arguments}
    {tab:earthAccelerationMessageArguments}
\endgroup

\begingroup
    \def\tempNameA{Quaternion W element}
    \def\tempNameB{Quaternion X element}
    \def\tempNameC{Quaternion Y element}
    \def\tempNameD{Quaternion Z element}
    \def\tempNameE{Earth acceleration X axis}
    \def\tempNameF{Earth acceleration Y axis}
    \def\tempNameG{Earth acceleration Z axis}
    \def\tempValueA{1}
    \def\tempValueB{0}
    \def\tempValueC{0}
    \def\tempValueD{0}
    \def\tempValueE{0}
    \def\tempValueF{0}
    \def\tempValueG{0}
    \def\tempAsciiFirst{E}
    \def\tempAsciiA{1.0000}
    \def\tempAsciiB{0.0000}
    \def\tempAsciiC{0.0000}
    \def\tempAsciiD{0.0000}
    \def\tempAsciiE{0.0000}
    \def\tempAsciiF{0.0000}
    \def\tempAsciiG{0.0000}
    \def\tempBinaryFirst{C5}
    \def\tempBinaryA{00 00 80 3F}
    \def\tempBinaryB{00 00 00 00}
    \def\tempBinaryC{00 00 00 00}
    \def\tempBinaryD{00 00 00 00}
    \def\tempBinaryE{00 00 00 00}
    \def\tempBinaryF{00 00 00 00}
    \def\tempBinaryG{00 00 00 00}
    \dataMessageExample
\endgroup

\subsubsection{\acs{AHRS} status message}

The \ac{AHRS} status message provides timestamped indications of the \ac{AHRS} status flags.  An \ac{AHRS} status message is sent each time a status flag changes.  \ac{AHRS} status messages are only sent if \ac{AHRS} messages are enabled in the device settings.  The first value of an \ac{ASCII} message is the character \enquote{U} and the arguments are four numerical values expressed to four decimal places.  The first byte of a binary message is 0xD5 (equal to 0x80 + \enquote{U}) and the arguments are four contiguous 32-bit floats.  The message arguments are described in \Fref{tab:ahrsStatusnMessageArguments}.

\begingroup
    \def\tempArgumentA{Initialising}
    \def\tempArgumentB{Angular rate recovery}
    \def\tempArgumentC{Acceleration recovery}
    \def\tempArgumentD{Magnetic recovery}
    \def\tempCaption{\acs{AHRS} status message arguments}
    \def\tempLabel{tab:ahrsStatusnMessageArguments}
    \dataMessageTable
    {\acs{AHRS} status message arguments (all arguments are Boolean, see \Fref{tab:booleanArgumentValues})}
    {tab:ahrsStatusnMessageArguments}
\endgroup

\customTable
{c l}
{Value & Boolean}
{
    0 & False\\
    !0 & True\\
}
{Boolean argument values}
{tab:booleanArgumentValues}

\begingroup
    \def\tempNameA{Initialising}
    \def\tempNameB{Angular rate recovery}
    \def\tempNameC{Acceleration recovery}
    \def\tempNameD{Magnetic recovery}
    \def\tempValueA{1}
    \def\tempValueB{0}
    \def\tempValueC{0}
    \def\tempValueD{0}
    \def\tempAsciiFirst{U}
    \def\tempAsciiA{1.0000}
    \def\tempAsciiB{0.0000}
    \def\tempAsciiC{0.0000}
    \def\tempAsciiD{0.0000}
    \def\tempBinaryFirst{D5}
    \def\tempBinaryA{00 00 80 3F}
    \def\tempBinaryB{00 00 00 00}
    \def\tempBinaryC{00 00 00 00}
    \def\tempBinaryD{00 00 00 00}
    \dataMessageExample
\endgroup

\subsubsection{High-g accelerometer message}

The High-g accelerometer message provides timestamped high-g accelerometer measurements.  High-g accelerometer messages are sent continuously at the message rate configured in the device settings.  The first value of an \ac{ASCII} message is the character \enquote{H} and the arguments are three numerical values expressed to four decimal places.  The first byte of a binary message is 0xC8 (equal to 0x80 + \enquote{H}) and the arguments are three contiguous 32-bit floats.  The message arguments are described in \Fref{tab:highGMessageArguments}.

\begingroup
    \def\tempArgumentA{High-g accelerometer X axis in g}
    \def\tempArgumentB{High-g accelerometer Y axis in g}
    \def\tempArgumentC{High-g accelerometer Z axis in g}
    \dataMessageTable
    {High-g accelerometer message arguments}
    {tab:highGMessageArguments}
\endgroup

\begingroup
    \def\tempNameA{High-g accelerometer X axis}
    \def\tempNameB{High-g accelerometer Y axis}
    \def\tempNameC{High-g accelerometer Z axis}
    \def\tempValueA{0}
    \def\tempValueB{0}
    \def\tempValueC{1}
    \def\tempAsciiFirst{H}
    \def\tempAsciiA{0.0000}
    \def\tempAsciiB{0.0000}
    \def\tempAsciiC{1.0000}
    \def\tempBinaryFirst{C8}
    \def\tempBinaryA{00 00 00 00}
    \def\tempBinaryB{00 00 00 00}
    \def\tempBinaryC{00 00 80 3F}
    \dataMessageExample
\endgroup

\subsubsection{Temperature message}

The temperature message provides timestamped temperature measurements.  Temperature messages are sent continuously at the message rate configured in the device settings.  The first value of an \ac{ASCII} message is the character \enquote{T} and the argument is a numerical value expressed to four decimal places.  The first byte of a binary message is 0xD4 (equal to 0x80 + \enquote{T}) and the argument is a 32-bit float.  The message arguments are described in \Fref{tab:temperatureMessageArguments}.

\begingroup
    \def\tempArgumentA{Temperature in degrees Celsius}
    \dataMessageTable
    {Temperature message arguments}
    {tab:temperatureMessageArguments}
\endgroup

\begingroup
    \def\tempNameA{Temperature}
    \def\tempValueA{25}
    \def\tempAsciiFirst{T}
    \def\tempAsciiA{25.0000}
    \def\tempBinaryFirst{D4}
    \def\tempBinaryA{00 00 41 C8}
    \dataMessageExample
\endgroup

\subsubsection{Battery message}

The battery message message provides timestamped measurements of the battery level and charger status.  Battery message messages are sent continuously at the message rate configured in the device settings.  The first value of an \ac{ASCII} message is the character \enquote{B} and the arguments are four numerical values expressed to four decimal places.  The first byte of a binary message is 0xC2 (equal to 0x80 + \enquote{B}) and the arguments are four contiguous 32-bit floats.  The message arguments are described in \Fref{tab:batteryMessageArguments}.

\begingroup
    \def\tempArgumentA{Battery percentage}
    \def\tempArgumentB{Battery voltage in volts}
    \def\tempArgumentC{Charging status (See \Fref{tab:chargingStatusEnumeration})}
    \dataMessageTable
    {Battery message arguments}
    {tab:batteryMessageArguments}
\endgroup

\customTable
{c l}
{Charging status & Description}
{
    0 & Not connected\\
    1 & Charging\\
    2 & Charging complete\\
}
{Charging status enumeration}
{tab:chargingStatusEnumeration}

\begingroup
    \def\tempNameA{Percentage}
    \def\tempNameB{Voltage}
    \def\tempNameC{Charging status}
    \def\tempValueA{100}
    \def\tempValueB{4.2}
    \def\tempValueC{2}
    \def\tempAsciiFirst{B}
    \def\tempAsciiA{100.0000}
    \def\tempAsciiB{4.2000}
    \def\tempAsciiC{2.0000}
    \def\tempBinaryFirst{C2}
    \def\tempBinaryA{00 00 C8 42}
    \def\tempBinaryB{66 66 86 40}
    \def\tempBinaryC{00 00 00 40}
    \dataMessageExample
\endgroup

\subsubsection{\acs{RSSI} message}

The \ac{RSSI} message provides timestamped Wi-Fi \ac{RSSI} measurements.  \ac{RSSI} messages are sent continuously at the message rate configured in the device settings.  \ac{RSSI} messages will only be sent if the x-IMU3 is connected in Wi-Fi client mode.  The first value of an \ac{ASCII} message is the character \enquote{W} and the arguments are two numerical values expressed to four decimal places.  The first byte of a binary message is 0xD7 (equal to 0x80 + \enquote{W}) and the arguments are two contiguous 32-bit floats.  The message arguments are described in \Fref{tab:rssiMessageArguments}.

\begingroup
    \def\tempArgumentA{\acs{RSSI} percentage}
    \def\tempArgumentB{\acs{RSSI} power in dBm}
    \dataMessageTable
    {\acs{RSSI} message arguments}
    {tab:rssiMessageArguments}
\endgroup

\begingroup
    \def\tempNameA{\acs{RSSI} percentage}
    \def\tempNameB{\acs{RSSI} power}
    \def\tempValueA{100}
    \def\tempValueB{-50}
    \def\tempAsciiFirst{W}
    \def\tempAsciiA{100.000}
    \def\tempAsciiB{-50.0000}
    \def\tempBinaryFirst{D7}
    \def\tempBinaryA{00 00 C8 42}
    \def\tempBinaryB{00 00 48 C2}
    \dataMessageExample
\endgroup

\subsubsection{Serial accessory message}

The serial accessory message provides timestamped received serial accessory data.  Serial accessory messages are sent as serial accessory data is received as configured in the device settings.  The first value of an \ac{ASCII} message is the character \enquote{S} and the argument is the received data.  Received byte values less than 0x20 or greater than 0x7E will be replaced with the character \enquote{?} so that the argument is a string of printable characters.  The string is not null-terminated.  The first byte of a binary message is 0xD3 (equal to 0x80 + \enquote{S}) and the argument is the unmodified received data.  The message arguments are described in \Fref{tab:serialAccessoryMessageArguments}.

\begingroup
    \def\tempArgumentA{Received serial accessory data}
    \dataMessageTable
    {Serial accessory message arguments}
    {tab:serialAccessoryMessageArguments}
\endgroup

\begingroup
    \def\tempNameA{Data}
    \def\tempValueA{0x61 0x62 0x63 0x31 0x32 0x33 0xF1 0xF2 0xF3}
    \def\tempAsciiFirst{S}
    \def\tempAsciiA{abc123???}
    \def\tempBinaryFirst{D3}
    \def\tempBinaryA{61 62 63 31 32 33 F1 F2 F3}
    \dataMessageExample
\endgroup

\subsubsection{Notification message}

The notification message provides timestamped notifications of system events.  Notification messages may be sent by the x-IMU3 at any time and cannot be disabled.  The first value of an \ac{ASCII} message is the character \enquote{N}.  The first byte of a binary message is 0xCE (equal to 0x80 + \enquote{N}).  The argument of both \ac{ASCII} and binary messages is a string of printable characters.  The string is not null-terminated.  The message arguments are described in \Fref{tab:notificationMessageArguments}.

\begingroup
    \def\tempArgumentA{Notification string}
    \dataMessageTable
    {Notification message arguments}
    {tab:notificationMessageArguments}
\endgroup

\begingroup
    \def\tempNameA{String}
    \def\tempValueA{This is a notification message.}
    \def\tempAsciiFirst{N}
    \def\tempAsciiA{This is a notification message.}
    \def\tempBinaryFirst{CE}
    \def\tempBinaryA{54 68 69 73 20 69 73 20 61 20 6E 6F 74 69 66 69 63 61 74 69 6F 6E 20 6D 65 73 73 61 67 65 2E}
    \dataMessageExample
\endgroup

\subsubsection{Error message}

The error message provides timestamped notifications of errors.  Error messages may be sent by the x-IMU3 at any time and cannot be disabled.  The first value of an \ac{ASCII} message is the character \enquote{F}.  The first byte of a binary message is 0xC6 (equal to 0x80 + \enquote{F}).  The argument of both \ac{ASCII} and binary messages is a string of printable characters.  The string is not null-terminated.  The message arguments are described in \Fref{tab:errorMessageArguments}.

\begingroup
    \def\tempArgumentA{Error string}
    \dataMessageTable
    {Notification message arguments}
    {tab:errorMessageArguments}
\endgroup

\begingroup
    \def\tempNameA{String}
    \def\tempValueA{This is an error message.}
    \def\tempAsciiFirst{F}
    \def\tempAsciiA{This is an error message.}
    \def\tempBinaryFirst{C6}
    \def\tempBinaryA{54 68 69 73 20 69 73 20 61 6E 20 65 72 72 6F 72 20 6D 65 73 73 61 67 65 2E}
    \dataMessageExample
\endgroup
