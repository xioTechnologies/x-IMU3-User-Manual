\section{Calibration}
\label{sec:calibration}

Each device is calibrated during production to achieve the specified accuracy.  The calibration process uses specialist equipment and propriety algorithms to calculate calibration parameters specific to each device.
 Calibration is performed at room temperature.  Accuracy will be reduced for operating temperatures that deviate from this temperature.  Please refer to the calibration certificate for specific temperature values.

\subsection{Inertial sensors}
\label{sec:inertialSensor}

The inertial sensors are the gyroscope, accelerometer, and high-g accelerometer.  Each inertial sensor is calibrated for axis sensitivity, axis offset, inter-axis misalignment, and package misalignment.  The inertial calibration model is described by \Fref{eq:inertial} where $i_c$ is the calibrated inertial measurement obtained from the uncalibrated inertial measurement, $i_u$, given the misalignment matrix, $M$, the sensitivity diagonal matrix, $s$, and the offset vector, $b$.  The inertial calibration model is expanded as \Fref{eq:inertialExpanded} to express the model as 15 scalar quantities.  The units of $i_c$, $i_u$, and $b$ are degrees per second for the gyroscope, and g for the accelerometer and high-g accelerometer.  $M$ and $s$ are ratios and therefore have no units.  The calibration parameters $M$, $s$, and $b$ for each inertial sensor can be accessed as device settings.

\begin{equation}
\label{eq:inertial}
i_c = M s (i_u - b)
\end{equation}

\begin{equation}
\label{eq:inertialExpanded}
    \begin{bmatrix}
        i_{cx}\\
        i_{cy}\\
        i_{cz}\\
    \end{bmatrix}
    =
    \begin{bmatrix}
        m_{xx} & m_{xy} & m_{xz}\\
        m_{yx} & m_{yy} & m_{yz}\\
        m_{zx} & m_{zy} & m_{zz}\\
    \end{bmatrix}
    \begin{bmatrix}
        s_{x} & 0 & 0\\
        0 & s_{y} & 0\\
        0 & 0 & s_{z}\\
    \end{bmatrix}
    \left(
    \begin{bmatrix}
        i_{ux}\\
        i_{uy}\\
        i_{uz}\\
    \end{bmatrix}
    -
    \begin{bmatrix}
        b_{x}\\
        b_{y}\\
        b_{z}\\
    \end{bmatrix}
    \right)
\end{equation}

\subsection{Magnetometer}
\label{sec:magnetometer}

The magnetometer is calibrated for soft iron and hard-iron characteristics.  Soft iron characteristics are distortions that alter the intensity and direction of the magnetic field measured by the magnetometer.  Soft iron calibration also accounts for magnetometer axis sensitivity, inter-axis misalignment, and package misalignment.  Hard iron characteristics are unintended magnetic fields generated by the device that offset magnetometer measurements.  Hard iron calibration also accounts for the magnetometer axis offset.

The magnetometer calibration model is described by \Fref{eq:magnetometer} where $m_c$ is the calibrated magnetometer measurement obtained from the uncalibrated magnetometer measurement, $m_u$, given the soft iron matrix, $S$, the hard iron vector, $h$.  The magnetometer calibration model is expanded as \Fref{eq:magnetometerExpanded} to express the model as 12 scalar quantities.  The units of $m_c$, $m_u$, and $h$ are \ac{a.u.}.  $S$ is a ratio and therefore has no units.  The calibration parameters $S$, $h$ can be accessed as device settings.

\begin{equation}
\label{eq:magnetometer}
m_c = S m_u - h
\end{equation}

\begin{equation}
\label{eq:magnetometerExpanded}
    \begin{bmatrix}
        m_{cx}\\
        m_{cy}\\
        m_{cz}\\
    \end{bmatrix}
    =
    \begin{bmatrix}
        s_{xx} & s_{xy} & s_{xz}\\
        s_{yx} & s_{yy} & s_{yz}\\
        s_{zx} & s_{zy} & s_{zz}\\
    \end{bmatrix}
    \begin{bmatrix}
        m_{ux}\\
        m_{uy}\\
        m_{uz}\\
    \end{bmatrix}
    -
    \begin{bmatrix}
        h_{x}\\
        h_{y}\\
        h_{z}\\
    \end{bmatrix}
\end{equation}

\subsection{Calibration certificate}

Each device is supplied with a calibration certificate.  The calibration certificate details all calibration parameters, the calibration date, the ambient temperature and device temperature during calibration, and any equipment used during the calibration process.  The certificate also includes graphs verifying the accuracy over the measurement range.  Calibration certificates are provided as a \ac{PDF} file.  There are three ways to access the calibration certificate for a device:

\begin{enumerate}[nolistsep]
    \item Scan the \ac{QR} code on the back of the device.
    \item Open the \enquote{Calibration Certificate.html} file stored on the \ac{microSD}.
    \item Enter the device serial number on the calibration certificate \href{https://x-io.co.uk/calibration-certificate/}{web page}.
\end{enumerate}
