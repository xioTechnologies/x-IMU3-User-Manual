% This file was generated by individual_settings.py

\begingroup
    \def\tempSection{Calibration date (read-only)}
    \def\tempLabel{sec:calibrationDate}
    \def\tempDescription
    {
        Calibration date.  \seeSection{sec:calibration}
    }
    \def\tempKey{calibrationDate}
    \def\tempType{string}
    \def\tempDefault{\enquote{Unknown}}
    \deviceSetting
\endgroup

\newcommand{\inertialMisalignmentDescription}[1] {#1 misalignment matrix (in row-major order) used for inertial sensor calibration.  \seeSection{sec:inertialSensor}}

\newcommand{\inertialSensitivityDescription}[1] {#1 sensitivity vector used for inertial sensor calibration.  \seeSection{sec:inertialSensor}}

\newcommand{\inertialOffsetDescription}[1] {#1 offset vector used for inertial sensor calibration.  \seeSection{sec:inertialSensor}}

\begingroup
    \def\tempSection{Gyroscope misalignment (read-only)}
    \def\tempLabel{sec:gyroscopeMisalignment}
    \def\tempDescription
    {
        \inertialMisalignmentDescription{Gyroscope}
    }
    \def\tempKey{gyroscopeMisalignment}
    \def\tempType{array of 9 numbers}
    \def\tempDefault{[1.0, 0.0, 0.0, 0.0, 1.0, 0.0, 0.0, 0.0, 1.0]}
    \deviceSetting
\endgroup

\begingroup
    \def\tempSection{Gyroscope sensitivity (read-only)}
    \def\tempLabel{sec:gyroscopeSensitivity}
    \def\tempDescription
    {
        \inertialSensitivityDescription{Gyroscope}
    }
    \def\tempKey{gyroscopeSensitivity}
    \def\tempType{array of 3 numbers}
    \def\tempDefault{[1.0, 1.0, 1.0]}
    \deviceSetting
\endgroup

\begingroup
    \def\tempSection{Gyroscope offset (read-only)}
    \def\tempLabel{sec:gyroscopeOffset}
    \def\tempDescription
    {
        \inertialOffsetDescription{Gyroscope}
    }
    \def\tempKey{gyroscopeOffset}
    \def\tempType{array of 3 numbers}
    \def\tempDefault{[0.0, 0.0, 0.0]}
    \deviceSetting
\endgroup

\begingroup
    \def\tempSection{Accelerometer misalignment (read-only)}
    \def\tempLabel{sec:accelerometerMisalignment}
    \def\tempDescription
    {
        \inertialMisalignmentDescription{Accelerometer}
    }
    \def\tempKey{accelerometerMisalignment}
    \def\tempType{array of 9 numbers}
    \def\tempDefault{[1.0, 0.0, 0.0, 0.0, 1.0, 0.0, 0.0, 0.0, 1.0]}
    \deviceSetting
\endgroup

\begingroup
    \def\tempSection{Accelerometer sensitivity (read-only)}
    \def\tempLabel{sec:accelerometerSensitivity}
    \def\tempDescription
    {
        \inertialSensitivityDescription{Accelerometer}
    }
    \def\tempKey{accelerometerSensitivity}
    \def\tempType{array of 3 numbers}
    \def\tempDefault{[1.0, 1.0, 1.0]}
    \deviceSetting
\endgroup

\begingroup
    \def\tempSection{Accelerometer offset (read-only)}
    \def\tempLabel{sec:accelerometerOffset}
    \def\tempDescription
    {
        \inertialOffsetDescription{Accelerometer}
    }
    \def\tempKey{accelerometerOffset}
    \def\tempType{array of 3 numbers}
    \def\tempDefault{[0.0, 0.0, 0.0]}
    \deviceSetting
\endgroup

\begingroup
    \def\tempSection{Soft iron matrix (read-only)}
    \def\tempLabel{sec:softIronMatrix}
    \def\tempDescription
    {
        Soft iron matrix (in row-major order) used for magnetometer calibration.  \seeSection{sec:magnetometer}
    }
    \def\tempKey{softIronMatrix}
    \def\tempType{array of 9 numbers}
    \def\tempDefault{[1.0, 0.0, 0.0, 0.0, 1.0, 0.0, 0.0, 0.0, 1.0]}
    \deviceSetting
\endgroup

\begingroup
    \def\tempSection{Hard iron offset (read-only)}
    \def\tempLabel{sec:hardIronOffset}
    \def\tempDescription
    {
        Hard iron offset vector used for magnetometer calibration.  \seeSection{sec:magnetometer}
    }
    \def\tempKey{hardIronOffset}
    \def\tempType{array of 3 numbers}
    \def\tempDefault{[0.0, 0.0, 0.0]}
    \deviceSetting
\endgroup

\begingroup
    \def\tempSection{High-g accelerometer misalignment (read-only)}
    \def\tempLabel{sec:highGAccelerometerMisalignment}
    \def\tempDescription
    {
        \inertialMisalignmentDescription{High-g accelerometer}
    }
    \def\tempKey{highGAccelerometerMisalignment}
    \def\tempType{array of 9 numbers}
    \def\tempDefault{[1.0, 0.0, 0.0, 0.0, 1.0, 0.0, 0.0, 0.0, 1.0]}
    \deviceSetting
\endgroup

\begingroup
    \def\tempSection{High-g accelerometer sensitivity (read-only)}
    \def\tempLabel{sec:highGAccelerometerSensitivity}
    \def\tempDescription
    {
        \inertialSensitivityDescription{High-g accelerometer}
    }
    \def\tempKey{highGAccelerometerSensitivity}
    \def\tempType{array of 3 numbers}
    \def\tempDefault{[1.0, 1.0, 1.0]}
    \deviceSetting
\endgroup

\begingroup
    \def\tempSection{High-g accelerometer offset (read-only)}
    \def\tempLabel{sec:highGAccelerometerOffset}
    \def\tempDescription
    {
        \inertialOffsetDescription{High-g accelerometer}
    }
    \def\tempKey{highGAccelerometerOffset}
    \def\tempType{array of 3 numbers}
    \def\tempDefault{[0.0, 0.0, 0.0]}
    \deviceSetting
\endgroup

\begingroup
    \def\tempSection{Device name}
    \def\tempLabel{sec:deviceName}
    \def\tempDescription
    {
        User-defined device name up to 31 characters long.
    }
    \def\tempKey{deviceName}
    \def\tempType{string}
    \def\tempDefault{\enquote{x-IMU3}}
    \deviceSetting
\endgroup

\begingroup
    \def\tempSection{Serial number (read-only)}
    \def\tempLabel{sec:serialNumber}
    \def\tempDescription
    {
        Unique 32-bit serial number expressed as a string of 8 hexadecimal digits in the format \enquote{XXXXXXXX}.
    }
    \def\tempKey{serialNumber}
    \def\tempType{string}
    \def\tempDefault{\enquote{Unknown}}
    \deviceSetting
\endgroup

\begingroup
    \def\tempSection{Firmware version (read-only)}
    \def\tempLabel{sec:firmwareVersion}
    \def\tempDescription
    {
        Firmware version.
    }
    \def\tempKey{firmwareVersion}
    \def\tempType{string}
    \def\tempDefault{\enquote{Unknown}}
    \deviceSetting
\endgroup

\begingroup
    \def\tempSection{Bootloader version (read-only)}
    \def\tempLabel{sec:bootloaderVersion}
    \def\tempDescription
    {
        Bootloader version.
    }
    \def\tempKey{bootloaderVersion}
    \def\tempType{string}
    \def\tempDefault{\enquote{Unknown}}
    \deviceSetting
\endgroup

\begingroup
    \def\tempSection{Hardware version (read-only)}
    \def\tempLabel{sec:hardwareVersion}
    \def\tempDescription
    {
        Hardware version.
    }
    \def\tempKey{hardwareVersion}
    \def\tempType{string}
    \def\tempDefault{\enquote{Unknown}}
    \deviceSetting
\endgroup

\begingroup
    \def\tempSection{Serial mode}
    \def\tempLabel{sec:serialMode}
    \def\tempDescription
    {
        Serial mode.
    }
    \def\tempKey{serialMode}
    \def\tempType{number}
    \def\tempDefault{0}
    \deviceSetting
\endgroup

\begingroup
    \def\tempSection{Serial baud rate}
    \def\tempLabel{sec:serialBaudRate}
    \def\tempDescription
    {
        Serial baud rate.
    }
    \def\tempKey{serialBaudRate}
    \def\tempType{number}
    \def\tempDefault{115200}
    \deviceSetting
\endgroup

\begingroup
    \def\tempSection{Serial \acs{RTS}/\acs{CTS} enabled}
    \def\tempLabel{sec:serialRtsCtsEnabled}
    \def\tempDescription
    {
        Serial \ac{RTS}/\ac{CTS} enabled.
    }
    \def\tempKey{serialRtsCtsEnabled}
    \def\tempType{true or false}
    \def\tempDefault{false}
    \deviceSetting
\endgroup

\begingroup
    \def\tempSection{Serial accessory number of bytes}
    \def\tempLabel{sec:serialAccessoryNumberOfBytes}
    \def\tempDescription
    {
        Serial accessory number of bytes.
    }
    \def\tempKey{serialAccessoryNumberOfBytes}
    \def\tempType{number}
    \def\tempDefault{1024}
    \deviceSetting
\endgroup

\begingroup
    \def\tempSection{Serial accessory termination byte}
    \def\tempLabel{sec:serialAccessoryTerminationByte}
    \def\tempDescription
    {
        Serial accessory termination byte.
    }
    \def\tempKey{serialAccessoryTerminationByte}
    \def\tempType{number}
    \def\tempDefault{10}
    \deviceSetting
\endgroup

\begingroup
    \def\tempSection{Serial accessory timeout}
    \def\tempLabel{sec:serialAccessoryTimeout}
    \def\tempDescription
    {
        Serial accessory timeout.
    }
    \def\tempKey{serialAccessoryTimeout}
    \def\tempType{number}
    \def\tempDefault{100}
    \deviceSetting
\endgroup

\begingroup
    \def\tempSection{Wireless mode}
    \def\tempLabel{sec:wirelessMode}
    \def\tempDescription
    {
        Configures the wireless mode.  The possible values are listed in \Fref{tab:wirelessModes}.  The current wireless mode is indicated by the \ac{LED} colour.  \seeSection{sec:led}

        \customTable
        {c l}
        {Value & Mode}
        {
            0 & Disabled\\
            1 & Wi-Fi client\\
            2 & Wi-Fi \acs{AP}\\
            3 & Bluetooth\\
        }
        {Wireless modes}
        {tab:wirelessModes}
    }
    \def\tempKey{wirelessMode}
    \def\tempType{number}
    \def\tempDefault{2}
    \deviceSetting
\endgroup

\begingroup
    \def\tempSection{Wireless firmware version (read-only)}
    \def\tempLabel{sec:wirelessFirmwareVersion}
    \def\tempDescription
    {
        Current wireless firmware version.
    }
    \def\tempKey{wirelessFirmwareVersion}
    \def\tempType{string}
    \def\tempDefault{\enquote{Unknown}}
    \deviceSetting
\endgroup

\begingroup
    \def\tempSection{External antennae enabled}
    \def\tempLabel{sec:externalAntennaeEnabled}
    \def\tempDescription
    {
        Enables (true) or disables (false) the external antennae.  An antennae must be connected to the U.FL connector if the external antennae is enabled.  The internal antennae will be used if the external antennae is disabled.
    }
    \def\tempKey{externalAntennaeEnabled}
    \def\tempType{true or false}
    \def\tempDefault{false}
    \deviceSetting
\endgroup

\begingroup
    \def\tempSection{Wi-Fi region}
    \def\tempLabel{sec:wiFiRegion}
    \def\tempDescription
    {
        Configures the region for Wi-Fi operation.  The x-IMU3 will operate according to the regulations of the selected region.  See \Fref{sec:wiFiClientSsid} for a list of channels available in each region.  The possible values are listed in \Fref{tab:wiFiRegions}.

        \customTable
        {c l}
        {Value & Region}
        {
            1 & \ac{US}\\
            2 & \ac{EU}\\
            3 & \ac{JP}\\
        }
        {Wi-Fi regions}
        {tab:wiFiRegions}
    }
    \def\tempKey{wiFiRegion}
    \def\tempType{number}
    \def\tempDefault{2}
    \deviceSetting
\endgroup

\begingroup
    \def\tempSection{Wi-Fi \acs{MAC} address (read-only)}
    \def\tempLabel{sec:wiFiMacAddress}
    \def\tempDescription
    {
        Wi-Fi \ac{MAC} address.
    }
    \def\tempKey{wiFiMacAddress}
    \def\tempType{string}
    \def\tempDefault{0}
    \deviceSetting
\endgroup

\begingroup
    \def\tempSection{Wi-Fi \acs{IP} address (read-only)}
    \def\tempLabel{sec:wiFiIPAddress}
    \def\tempDescription
    {
        Current \ac{IP} address of the x-IMU3.  A value of 0.0.0.0 indicates that the x-IMU3 does not yet have an \ac{IP} address.
    }
    \def\tempKey{wiFiIPAddress}
    \def\tempType{string}
    \def\tempDefault{0}
    \deviceSetting
\endgroup

\begingroup
    \def\tempSection{Wi-Fi client \acs{SSID}}
    \def\tempLabel{sec:wiFiClientSsid}
    \def\tempDescription
    {
        Configures the \ac{SSID} in Wi-Fi client mode.  This is the name of the router that the x-IMU3 will connect to.  The \ac{SSID} may be up to 31 characters long.  If the \ac{SSID} is left blank then the x-IMU3 will use the \ac{SSID} \enquote{x-IMU3 Network}.
    }
    \def\tempKey{wiFiClientSsid}
    \def\tempType{string}
    \def\tempDefault{\enquote{}}
    \deviceSetting
\endgroup

\begingroup
    \def\tempSection{Wi-Fi client key}
    \def\tempLabel{sec:wiFiClientKey}
    \def\tempDescription
    {
        Configures the security key in Wi-Fi client mode.  This is the password for the router that the x-IMU3 will connect to.  If the router does not require a password then this setting will be ignored.  The key must be between 8 and 31 characters long.  If the key is left blank then the x-IMU3 will use the key \enquote{xiotechnologies}.
    }
    \def\tempKey{wiFiClientKey}
    \def\tempType{string}
    \def\tempDefault{\enquote{}}
    \deviceSetting
\endgroup

\begingroup
    \def\tempSection{Wi-Fi client channel}
    \def\tempLabel{sec:wiFiClientChannel}
    \def\tempDescription
    {
        Configures the channel in Wi-Fi client mode.  This is the channel of the router that the x-IMU3 will connect to.  Setting the correct channel will decrease the time taken to connect.  If the channel is incorrect then the x-IMU3 will search all channels and then update this setting for the correct channel.  The possible channels are listed in \Fref{tab:wiFiClientChannels}.

        \customTable
        {c c l}
        {Channel & Band & Notes}
        {
            0 & - & All channels\\
            1, 2, 3, 4, 5, 6, 7, 8, 9, 10, 11 & 2.4 GHz & -\\
            12, 13 & 2.4 GHz & Invalid for \acs{US}\\
            14 & 2.4 GHz & Invalid for \acs{US} and \acs{EU}\\
            36, 40, 44, 48 & 5 GHz & -\\
            52, 56, 60, 64, 100, 104, 108, 112, 116 & 5 GHz, \acs{DFS} & -\\
            120, 124, 128 & 5 GHz, \acs{DFS} & Invalid for \acs{US}\\
            132, 136, 140 & 5 GHz, \acs{DFS} & -\\
            149, 153, 157, 161, 165 & 5 GHz & Invalid for \acs{EU} and \acs{JP}\\
        }
        {Wi-Fi client channels}
        {tab:wiFiClientChannels}
    }
    \def\tempKey{wiFiClientChannel}
    \def\tempType{number}
    \def\tempDefault{0}
    \deviceSetting
\endgroup

\begingroup
    \def\tempSection{Wi-Fi client \acs{DHCP} enabled}
    \def\tempLabel{sec:wiFiClientDhcpEnabled}
    \def\tempDescription
    {
        Enables (true) or disables (false) \ac{DHCP} in Wi-Fi client mode.  If \ac{DHCP} is enabled then the x-IMU3 will be assigned an \ac{IP} address by the router.  This is an automatic process and the recommend mode of operation.  If \ac{DHCP} is disabled then the \ac{IP} address, netmask, and gateway must be configured by the user.
    }
    \def\tempKey{wiFiClientDhcpEnabled}
    \def\tempType{true or false}
    \def\tempDefault{true}
    \deviceSetting
\endgroup

\begingroup
    \def\tempSection{Wi-Fi client \acs{IP} address}
    \def\tempLabel{sec:wiFiClientIPAddress}
    \def\tempDescription
    {
        Configures the x-IMU3 \ac{IP} address in Wi-Fi client mode when \ac{DHCP} is disabled.  This setting will be ignored if \ac{DHCP} is enabled.
    }
    \def\tempKey{wiFiClientIPAddress}
    \def\tempType{string}
    \def\tempDefault{\enquote{192.168.1.2}}
    \deviceSetting
\endgroup

\begingroup
    \def\tempSection{Wi-Fi client netmask}
    \def\tempLabel{sec:wiFiClientNetmask}
    \def\tempDescription
    {
        Configures the netmask in Wi-Fi client mode when \ac{DHCP} is disabled.  This setting will be ignored if \ac{DHCP} is enabled.
    }
    \def\tempKey{wiFiClientNetmask}
    \def\tempType{string}
    \def\tempDefault{\enquote{255.255.255.0}}
    \deviceSetting
\endgroup

\begingroup
    \def\tempSection{Wi-Fi client gateway}
    \def\tempLabel{sec:wiFiClientGateway}
    \def\tempDescription
    {
        Configures the gateway in Wi-Fi client mode when \ac{DHCP} is disabled.  This setting will be ignored if \ac{DHCP} is enabled.
    }
    \def\tempKey{wiFiClientGateway}
    \def\tempType{string}
    \def\tempDefault{\enquote{192.168.1.1}}
    \deviceSetting
\endgroup

\begingroup
    \def\tempSection{Wi-Fi \acs{AP} \acs{SSID}}
    \def\tempLabel{sec:wiFiAPSsid}
    \def\tempDescription
    {
		Configures the \ac{SSID} of the x-IMU3 in Wi-Fi \ac{AP} mode.  This is the name of the network created by the x-IMU3.  The \ac{SSID} may be up to 31 characters long.  If the \ac{SSID} is left blank then the x-IMU3 will use the \ac{SSID} \enquote{x-IMU3}.
    }
    \def\tempKey{wiFiAPSsid}
    \def\tempType{string}
    \def\tempDefault{\enquote{}}
    \deviceSetting
\endgroup

\begingroup
    \def\tempSection{Wi-Fi \acs{AP} key}
    \def\tempLabel{sec:wiFiAPKey}
    \def\tempDescription
    {
		Configures the security key in Wi-Fi \ac{AP} mode.  This is the password required to connect to the network created by the x-IMU3.  The key must be between 8 and 31 characters long.  If the key is left blank then the network will not use security and a password would not be required to connect to the x-IMU3.
    }
    \def\tempKey{wiFiAPKey}
    \def\tempType{string}
    \def\tempDefault{\enquote{}}
    \deviceSetting
\endgroup

\begingroup
    \def\tempSection{Wi-Fi \acs{AP} channel}
    \def\tempLabel{sec:wiFiAPChannel}
    \def\tempDescription
    {
        Configures the channel of the x-IMU3 in Wi-Fi \ac{AP} mode.  This is the channel of the network created by the x-IMU3.  The possible channels are listed in \Fref{tab:wiFiAPChannels}.

        \customTable
        {c c l}
        {Channel & Band & Notes}
        {
            1, 2, 3, 4, 5, 6, 7, 8, 9, 10, 11 & 2.4 GHz & -\\
            12, 13 & 2.4 GHz & Invalid for \acs{US}\\
            14 & 2.4 GHz & Invalid for \acs{US} and \acs{EU}\\
            36, 40, 44, 48 & 5 GHz & -\\
            149, 153, 157, 161, 165 & 5 GHz & Invalid for \acs{EU} and \acs{JP}\\
        }
        {Wi-Fi \acs{AP} channels}
        {tab:wiFiAPChannels}
    }
    \def\tempKey{wiFiAPChannel}
    \def\tempType{number}
    \def\tempDefault{36}
    \deviceSetting
\endgroup

\begingroup
    \def\tempSection{Wi-Fi \acs{AP} \acs{IP} address}
    \def\tempLabel{sec:wiFiAPIPAddress}
    \def\tempDescription
    {
        Configures the x-IMU3 \ac{IP} address in Wi-Fi \ac{AP} mode
    }
    \def\tempKey{wiFiAPIPAddress}
    \def\tempType{string}
    \def\tempDefault{\enquote{192.169.1.1}}
    \deviceSetting
\endgroup

\begingroup
    \def\tempSection{\acs{TCP} port}
    \def\tempLabel{sec:tcpPort}
    \def\tempDescription
    {
        \ac{TCP} port.
    }
    \def\tempKey{tcpPort}
    \def\tempType{number}
    \def\tempDefault{7000}
    \deviceSetting
\endgroup

\begingroup
    \def\tempSection{\acs{UDP} \acs{IP} address}
    \def\tempLabel{sec:udpIPAddress}
    \def\tempDescription
    {
        Configures the \ac{IP} address that the x-IMU3 will send to.  This should be the \ac{IP} address of the host.  If the \ac{IP} address is \enquote{0.0.0.0} then the x-IMU3 will send to the source \ac{IP} address of the most recent \ac{UDP} packet received by the x-IMU3.
    }
    \def\tempKey{udpIPAddress}
    \def\tempType{string}
    \def\tempDefault{\enquote{0.0.0.0}}
    \deviceSetting
\endgroup

\begingroup
    \def\tempSection{\acs{UDP} send port}
    \def\tempLabel{sec:udpSendPort}
    \def\tempDescription
    {
        Configures the \ac{UDP} port that the x-IMU3 will send to.  This is the port that a host should listen to.  If the port is 0 then the x-IMU3 will send to the source port of the most recent \ac{UDP} packet received by the x-IMU3.
    }
    \def\tempKey{udpSendPort}
    \def\tempType{number}
    \def\tempDefault{0}
    \deviceSetting
\endgroup

\begingroup
    \def\tempSection{\acs{UDP} receive port}
    \def\tempLabel{sec:udpReceivePort}
    \def\tempDescription
    {
        Configures the \ac{UDP} port that the x-IMU3 will listen to.  This is the port that a host should send to.
    }
    \def\tempKey{udpReceivePort}
    \def\tempType{number}
    \def\tempDefault{9000}
    \deviceSetting
\endgroup

\begingroup
    \def\tempSection{\acs{UDP} low latency}
    \def\tempLabel{sec:udpLowLatency}
    \def\tempDescription
    {
        Enables (true) or disables (false) low latency \ac{UDP} performance.  Enabling low latency \ac{UDP} performance will typically result in each data message being sent as a single \ac{UDP} packet.  This provides the best possible latency performance but will significantly limit the maximum throughput and number of devices able to stream on the same network.  It is recommended that low latency \ac{UDP} performance is disabled for most applications.
    }
    \def\tempKey{udpLowLatency}
    \def\tempType{true or false}
    \def\tempDefault{false}
    \deviceSetting
\endgroup

\begingroup
    \def\tempSection{Synchronisation enabled}
    \def\tempLabel{sec:synchronisationEnabled}
    \def\tempDescription
    {
        Enables (true) or disables (false) synchronisation.  \seeSection{sec:synchronisation}
    }
    \def\tempKey{synchronisationEnabled}
    \def\tempType{true or false}
    \def\tempDefault{true}
    \deviceSetting
\endgroup

\begingroup
    \def\tempSection{Synchronisation network latency}
    \def\tempLabel{sec:synchronisationNetworkLatency}
    \def\tempDescription
    {
        The network latency (in microseconds) used by the synchronisation algorithm.
    }
    \def\tempKey{synchronisationNetworkLatency}
    \def\tempType{number}
    \def\tempDefault{1500}
    \deviceSetting
\endgroup

\begingroup
    \def\tempSection{Bluetooth address (read-only)}
    \def\tempLabel{sec:bluetoothAddress}
    \def\tempDescription
    {
        x-IMU3 Bluetooth address.
    }
    \def\tempKey{bluetoothAddress}
    \def\tempType{number}
    \def\tempDefault{0}
    \deviceSetting
\endgroup

\begingroup
    \def\tempSection{Bluetooth name}
    \def\tempLabel{sec:bluetoothName}
    \def\tempDescription
    {
        Configures the name of the x-IMU3 in Bluetooth mode. The name may be up to 31 characters long. If the name is left blank then the x-IMU3 will use the name \enquote{x-IMU3}.
    }
    \def\tempKey{bluetoothName}
    \def\tempType{string}
    \def\tempDefault{\enquote{}}
    \deviceSetting
\endgroup

\begingroup
    \def\tempSection{Bluetooth pin code}
    \def\tempLabel{sec:bluetoothPinCode}
    \def\tempDescription
    {
        Configures the Bluetooth pin code.  The pin code may be up to 15 characters long.  If the pin code is left blank then the x-IMU3 will use the name \enquote{1234}.
    }
    \def\tempKey{bluetoothPinCode}
    \def\tempType{string}
    \def\tempDefault{\enquote{}}
    \deviceSetting
\endgroup

\begingroup
    \def\tempSection{Bluetooth discovery mode}
    \def\tempLabel{sec:bluetoothDiscoveryMode}
    \def\tempDescription
    {
        Configures the Bluetooth discovery mode.  The possible values are listed in \Fref{tab:bluetoothDiscoveryModes}.

        \customTable
        {c l}
        {Value & Mode}
        {
            0 & Disabled\\
            1 & Enabled\\
            2 & Limited\\
        }
        {Bluetooth discovery modes}
        {tab:bluetoothDiscoveryModes}
    }
    \def\tempKey{bluetoothDiscoveryMode}
    \def\tempType{number}
    \def\tempDefault{2}
    \deviceSetting
\endgroup

\begingroup
    \def\tempSection{Bluetooth paired address (read-only)}
    \def\tempLabel{sec:bluetoothPairedAddress}
    \def\tempDescription
    {
        Bluetooth address of the device paired with the x-IMU3.
    }
    \def\tempKey{bluetoothPairedAddress}
    \def\tempType{number}
    \def\tempDefault{0}
    \deviceSetting
\endgroup

\begingroup
    \def\tempSection{Bluetooth paired link key (read-only)}
    \def\tempLabel{sec:bluetoothPairedLinkKey}
    \def\tempDescription
    {
        Bluetooth link key of device paired with the x-IMU3.
    }
    \def\tempKey{bluetoothPairedLinkKey}
    \def\tempType{number}
    \def\tempDefault{0}
    \deviceSetting
\endgroup

\begingroup
    \def\tempSection{Data logger enabled}
    \def\tempLabel{sec:dataLoggerEnabled}
    \def\tempDescription
    {
        Enables (true) or disables (false) the data logger.  \seeSection{sec:dataLogger}
    }
    \def\tempKey{dataLoggerEnabled}
    \def\tempType{true or false}
    \def\tempDefault{false}
    \deviceSetting
\endgroup

\begingroup
    \def\tempSection{Data logger file name prefix}
    \def\tempLabel{sec:dataLoggerFileNamePrefix}
    \def\tempDescription
    {
        Configures the prefix part of the file name used by the data logger.  Non-alphanumeric characters will not be included in the file name.  \seeSection{sec:fileName}
    }
    \def\tempKey{dataLoggerFileNamePrefix}
    \def\tempType{string}
    \def\tempDefault{\enquote{}}
    \deviceSetting
\endgroup

\begingroup
    \def\tempSection{Data logger file name time enabled}
    \def\tempLabel{sec:dataLoggerFileNameTimeEnabled}
    \def\tempDescription
    {
        Enables (true) or disables (false) the time part of the file name used by the data logger.  \seeSection{sec:fileName}
    }
    \def\tempKey{dataLoggerFileNameTimeEnabled}
    \def\tempType{true or false}
    \def\tempDefault{true}
    \deviceSetting
\endgroup

\begingroup
    \def\tempSection{Data logger file name counter enabled}
    \def\tempLabel{sec:dataLoggerFileNameCounterEnabled}
    \def\tempDescription
    {
        Enables (true) or disables (false) the counter part of the file name used by the data logger.  \seeSection{sec:fileName}
    }
    \def\tempKey{dataLoggerFileNameCounterEnabled}
    \def\tempType{true or false}
    \def\tempDefault{false}
    \deviceSetting
\endgroup

\begingroup
    \def\tempSection{Data logger max file size}
    \def\tempLabel{sec:dataLoggerMaxFileSize}
    \def\tempDescription
    {
        Configures the the maximum size (in MB) of files created by the data logger.  A value of zero will configure the maximum file size supported by \ac{FAT32} (4 GB).  \seeSection{sec:maximumFileSizeAndPeriod}
    }
    \def\tempKey{dataLoggerMaxFileSize}
    \def\tempType{number}
    \def\tempDefault{0}
    \deviceSetting
\endgroup

\begingroup
    \def\tempSection{Data logger max file period}
    \def\tempLabel{sec:dataLoggerMaxFilePeriod}
    \def\tempDescription
    {
        Configures the the maximum period (in seconds) of files created by the data logger.  A value of zero will configure an unlimited period.  \seeSection{sec:maximumFileSizeAndPeriod}
    }
    \def\tempKey{dataLoggerMaxFilePeriod}
    \def\tempType{number}
    \def\tempDefault{0}
    \deviceSetting
\endgroup

\begingroup
    \def\tempSection{Axes alignment}
    \def\tempLabel{sec:axesAlignment}
    \def\tempDescription
    {
        Axes alignment describing the sensor axes relative to the body axes. For example, if the body X axis is aligned with the sensor Y axis and the body Y axis is aligned with sensor X axis but pointing the opposite direction then alignment is +Y-X+Z.  The possible values are listed in \Fref{tab:axesAlignments}.

        \customTable
        {c l}
        {Value & Axes alignment}
        {
            0 & +X+Y+Z\\
            1 & +X-Z+Y\\
            2 & +X-Y-Z\\
            3 & +X+Z-Y\\
            4 & -X+Y-Z\\
            5 & -X+Z+Y\\
            6 & -X-Y+Z\\
            7 & -X-Z-Y\\
            8 & +Y-X+Z\\
            9 & +Y-Z-X\\
            10 & +Y+X-Z\\
            11 & +Y+Z+X\\
            12 & -Y+X+Z\\
            13 & -Y-Z+X\\
            14 & -Y-X-Z\\
            15 & -Y+Z-X\\
            16 & +Z+Y-X\\
            17 & +Z+X+Y\\
            18 & +Z-Y+X\\
            19 & +Z-X-Y\\
            20 & -Z+Y+X\\
            21 & -Z-X+Y\\
            22 & -Z-Y-X\\
            23 & -Z+X-Y\\
        }
        {Axes alignments}
        {tab:axesAlignments}
    }
    \def\tempKey{axesAlignment}
    \def\tempType{number}
    \def\tempDefault{0}
    \deviceSetting
\endgroup

\begingroup
    \def\tempSection{Gyroscope offset correction enabled}
    \def\tempLabel{sec:gyroscopeOffsetCorrectionEnabled}
    \def\tempDescription
    {
        Gyroscope offset correction enabled.
    }
    \def\tempKey{gyroscopeOffsetCorrectionEnabled}
    \def\tempType{true or false}
    \def\tempDefault{true}
    \deviceSetting
\endgroup

\begingroup
    \def\tempSection{\acs{AHRS} axes convention}
    \def\tempLabel{sec:ahrsAxesConvention}
    \def\tempDescription
    {
        Configures  the Earth axes convention used by the \ac{AHRS} algorithm.  The possible values are listed in \Fref{tab:axesConventions}.
        \customTable
        {c l}
        {Value & Axes Convention}
        {
            0 & North-West-Up (NWU)\\
            1 & East-North-Up (ENU)\\
            2 & North-East-Down (NED)\\
        }
        {Axes conventions}
        {tab:axesConventions}
    }
    \def\tempKey{ahrsAxesConvention}
    \def\tempType{number}
    \def\tempDefault{0}
    \deviceSetting
\endgroup

\begingroup
    \def\tempSection{\acs{AHRS} gain}
    \def\tempLabel{sec:ahrsGain}
    \def\tempDescription
    {
        \ac{AHRS} gain.
    }
    \def\tempKey{ahrsGain}
    \def\tempType{number}
    \def\tempDefault{0.5}
    \deviceSetting
\endgroup

\begingroup
    \def\tempSection{\acs{AHRS} ignore magnetometer}
    \def\tempLabel{sec:ahrsIgnoreMagnetometer}
    \def\tempDescription
    {
        \ac{AHRS} ignore magnetometer.
    }
    \def\tempKey{ahrsIgnoreMagnetometer}
    \def\tempType{true or false}
    \def\tempDefault{false}
    \deviceSetting
\endgroup

\begingroup
    \def\tempSection{\acs{AHRS} acceleration rejection enabled}
    \def\tempLabel{sec:ahrsAccelerationRejectionEnabled}
    \def\tempDescription
    {
        \ac{AHRS} acceleration rejection enabled.
    }
    \def\tempKey{ahrsAccelerationRejectionEnabled}
    \def\tempType{true or false}
    \def\tempDefault{true}
    \deviceSetting
\endgroup

\begingroup
    \def\tempSection{\acs{AHRS} magnetic rejection enabled}
    \def\tempLabel{sec:ahrsMagneticRejectionEnabled}
    \def\tempDescription
    {
        \ac{AHRS} magnetic rejection enabled.
    }
    \def\tempKey{ahrsMagneticRejectionEnabled}
    \def\tempType{true or false}
    \def\tempDefault{true}
    \deviceSetting
\endgroup

\begingroup
    \def\tempSection{Binary mode enabled}
    \def\tempLabel{sec:binaryModeEnabled}
    \def\tempDescription
    {
		Enables (true) or disables (false) binary data messages.  If binary data messages are disabled then data messages will be \ac{ASCII}.  \seeSection{sec:dataMessages}
    }
    \def\tempKey{binaryModeEnabled}
    \def\tempType{true or false}
    \def\tempDefault{true}
    \deviceSetting
\endgroup

\begingroup
    \def\tempSection{\acs{USB} data messages enabled}
    \def\tempLabel{sec:usbDataMessagesEnabled}
    \def\tempDescription
    {
        Enables (true) or disables (false) the sending of data messages for the \ac{USB} communication interface.  The sending of notification and error messages cannot be disabled.
    }
    \def\tempKey{usbDataMessagesEnabled}
    \def\tempType{true or false}
    \def\tempDefault{true}
    \deviceSetting
\endgroup

\begingroup
    \def\tempSection{Serial data messages enabled}
    \def\tempLabel{sec:serialDataMessagesEnabled}
    \def\tempDescription
    {
        Enables (true) or disables (false) the sending of data messages for the serial communication interface when it is in normal mode.  The sending of notification and error messages cannot be disabled.
    }
    \def\tempKey{serialDataMessagesEnabled}
    \def\tempType{true or false}
    \def\tempDefault{true}
    \deviceSetting
\endgroup

\begingroup
    \def\tempSection{\acs{TCP} data messages enabled}
    \def\tempLabel{sec:tcpDataMessagesEnabled}
    \def\tempDescription
    {
        Enables (true) or disables (false) the sending of data messages for the \ac{TCP} communication interface.  The sending of notification and error messages cannot be disabled.
    }
    \def\tempKey{tcpDataMessagesEnabled}
    \def\tempType{true or false}
    \def\tempDefault{true}
    \deviceSetting
\endgroup

\begingroup
    \def\tempSection{\acs{UDP} data messages enabled}
    \def\tempLabel{sec:udpDataMessagesEnabled}
    \def\tempDescription
    {
        Enables (true) or disables (false) the sending of data messages for the \ac{UDP} communication interface.  The sending of notification and error messages cannot be disabled.
    }
    \def\tempKey{udpDataMessagesEnabled}
    \def\tempType{true or false}
    \def\tempDefault{true}
    \deviceSetting
\endgroup

\begingroup
    \def\tempSection{Bluetooth data messages enabled}
    \def\tempLabel{sec:bluetoothDataMessagesEnabled}
    \def\tempDescription
    {
        Bluetooth data messages enabled.
    }
    \def\tempKey{bluetoothDataMessagesEnabled}
    \def\tempType{true or false}
    \def\tempDefault{true}
    \deviceSetting
\endgroup

\begingroup
    \def\tempSection{Data logger data messages enabled}
    \def\tempLabel{sec:dataLoggerDataMessagesEnabled}
    \def\tempDescription
    {
        Enables (true) or disables (false) the sending of data messages for the on-board data logging.  The sending of notification and error messages cannot be disabled.
    }
    \def\tempKey{dataLoggerDataMessagesEnabled}
    \def\tempType{true or false}
    \def\tempDefault{true}
    \deviceSetting
\endgroup

\begingroup
    \def\tempSection{\acs{AHRS} message type}
    \def\tempLabel{sec:ahrsMessageType}
    \def\tempDescription
    {
        Configures the \ac{AHRS} message type.  The possible values are listed in \Fref{tab:ahrsMessageTypes}.

        \customTable
        {c l}
        {Value & Message type}
        {
            0 & Quaternion\\
            1 & Rotation matrix\\
            2 & Euler angles\\
            3 & Linear acceleration\\
            4 & Earth acceleration\\
        }
        {\ac{AHRS} message types}
        {tab:ahrsMessageTypes}
    }
    \def\tempKey{ahrsMessageType}
    \def\tempType{number}
    \def\tempDefault{0}
    \deviceSetting
\endgroup

\newcommand{\messageRateDivisorDescription}[6]
{
    Configures the #1 message rate as the fixed sample of #2 Hz divided by the the message rate divisor.  A message rate divisor of zero will disable the messages.  Example message rates are listed in \Fref{#6}.  See \Fref{sec:sampleRatesMessageRatesAndTimestamps} for more information about messages rates.

    \customTable
    {c c}
    {Divisor & Message rate}
    {
        0 & Disabled\\
        1 & #2 messages/s\\
        2 & #3 messages/s\\
        3 & #4 messages/s\\
        4 & #5 messages/s\\
        ... & ...\\
    }
    {Example #1 message rates}
    {#6}
}

\begingroup
    \def\tempSection{Inertial message rate divisor}
    \def\tempLabel{sec:inertialMessageRateDivisor}
    \def\tempDescription
    {
        \messageRateDivisorDescription
        {inertial}
        {400}
        {200}
        {133.33}
        {100}
        {tab:exampleInertialMessageRates}
    }
    \def\tempKey{inertialMessageRateDivisor}
    \def\tempType{number}
    \def\tempDefault{8}
    \deviceSetting
\endgroup

\begingroup
    \def\tempSection{Magnetometer message rate divisor}
    \def\tempLabel{sec:magnetometerMessageRateDivisor}
    \def\tempDescription
    {
        \messageRateDivisorDescription
        {magnetometer}
        {20}
        {10}
        {6.67}
        {5}
        {tab:exampleMagnetometerMessageRates}
    }
    \def\tempKey{magnetometerMessageRateDivisor}
    \def\tempType{number}
    \def\tempDefault{1}
    \deviceSetting
\endgroup

\begingroup
    \def\tempSection{\acs{AHRS} message rate divisor}
    \def\tempLabel{sec:ahrsMessageRateDivisor}
    \def\tempDescription
    {
        \messageRateDivisorDescription
        {\ac{AHRS}}
        {400}
        {200}
        {133.33}
        {100}
        {tab:exampleAhrsrMessageRates}
    }
    \def\tempKey{ahrsMessageRateDivisor}
    \def\tempType{number}
    \def\tempDefault{8}
    \deviceSetting
\endgroup

\begingroup
    \def\tempSection{High-g accelerometer message rate divisor}
    \def\tempLabel{sec:highGAccelerometerMessageRateDivisor}
    \def\tempDescription
    {
        \messageRateDivisorDescription
        {high-g accelerometer}
        {1600}
        {800}
        {533.33}
        {400}
        {tab:exampleHighGAccelerometerMessageRates}
    }
    \def\tempKey{highGAccelerometerMessageRateDivisor}
    \def\tempType{number}
    \def\tempDefault{32}
    \deviceSetting
\endgroup

\begingroup
    \def\tempSection{Temperature message rate divisor}
    \def\tempLabel{sec:temperatureMessageRateDivisor}
    \def\tempDescription
    {
        \messageRateDivisorDescription
        {temperature}
        {5}
        {2.5}
        {1.67}
        {1.25}
        {tab:exampleTemperatureMessageRates}
    }
    \def\tempKey{temperatureMessageRateDivisor}
    \def\tempType{number}
    \def\tempDefault{5}
    \deviceSetting
\endgroup

\begingroup
    \def\tempSection{Battery message rate divisor}
    \def\tempLabel{sec:batteryMessageRateDivisor}
    \def\tempDescription
    {
        \messageRateDivisorDescription
        {battery}
        {5}
        {2.5}
        {1.67}
        {1.25}
        {tab:exampleBatteryMessageRates}
    }
    \def\tempKey{batteryMessageRateDivisor}
    \def\tempType{number}
    \def\tempDefault{5}
    \deviceSetting
\endgroup

\begingroup
    \def\tempSection{\acs{RSSI} message rate divisor}
    \def\tempLabel{sec:rssiMessageRateDivisor}
    \def\tempDescription
    {
        \messageRateDivisorDescription
        {\ac{RSSI}}
        {1}
        {0.5}
        {0.33}
        {0.25}
        {tab:exampleRssiMessageRates}
    }
    \def\tempKey{rssiMessageRateDivisor}
    \def\tempType{number}
    \def\tempDefault{1}
    \deviceSetting
\endgroup
